\documentclass{beamer}
%\documentclass[aspectratio=169]{beamer}
\usepackage[utf8]{inputenc}
\usepackage{lipsum}
\usetheme{ubb}

\title{Revisión bibliográfica de los métodos de cálculo de rendimiento  de maquinaria pesada.}


\author[Barrales, A.]
{Alex Barrales Araneda}


\institute[]
{
  %\inst{1}%
  Departamento de Ingeniería Industrial\\
  Universidad del Bío-Bío
}

\date[05/08/2020]{05 de agosto de 2020. Concepción, Chile}

\renewcommand\advisor
{Carlos Obreque Niñez}

\renewcommand\committee
{Patricio Álvarez\\
 Patricio Álvarez}

\begin{document}

\begin{frame}
\titlepage
\end{frame}


\begin{frame} 
\frametitle{Título 1} 
\framesubtitle{Subtítulo 1} 
\begin{theorem}
\lipsum[0]
\end{theorem}
\begin{enumerate} 
\item<1-| alert@1> Suppose $p$ were the largest prime number. 
\item<2-> Let $q$ be the product of the first $p$ numbers. 
\item<3-> Then $q+1$ is not divisible by any of them. 
\item<1-> But $q + 1$ is greater than $1$, thus divisible by some prime
number not in the first $p$ numbers.
\end{enumerate}
\end{frame}

\begin{frame}{A longer title}
\begin{itemize}
\item one
\item two
\end{itemize}
\end{frame}

\end{document}
